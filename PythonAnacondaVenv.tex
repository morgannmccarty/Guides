\documentclass{article}

\usepackage{hyperref}

\title{Python, Anaconda, and Virtual Environments}
\author{Morgan McCarty}
\date{22 February 2023}

\begin{document}
    \maketitle

    \section{Installing Python and/or Anaconda}

        \subsection{Download and Install}
            https://www.anaconda.com/products/distribution \\Download corresponding version of Anaconda for your operating system.
            \\Follow the instructions using recommended settings

        \subsection{Adding to Path}
            This step will vary significantly for each OS

            \subsubsection{What is Path?}
                Path is an environmental variable which allows programs to be run from the command line/terminal without being directly accessed in the current folder (or without a directory being provided).
                This means that any program or folder of programs added to Path will function with just their command entered into any terminal.
            
            \subsubsection{Benefits of Adding Programs to Path}
            There are many benefits to doing this - notably you can execute programs with considerably less setup and it makes detection of system commands (built by those programs) much easier

            \subsubsection{Risks of Adding Programs to Path}
            There are system commands that, if overridden, can severely mess up your computer if they can no longer be executed (thankfully this is easily reversible)

\end{document}

